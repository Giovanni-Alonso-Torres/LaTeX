\chapter{\label{cap:introduccion}Introducción}

Diversas áreas de la Física pretenden explicar la materia y la radiación a nivel fundamental, ya sea en el régimen de altas o bajas energías. Ambas áreas buscan nuevos estados o interacciones que arrojen pistas para comprender la Naturaleza, en el estudio a bajas energías es posible utilizar la materia para producir estados no clásicos de luz. Dicha radiación exhibe comportamientos que no se pueden explicar con teorías clásicas, cuando este tipo de situaciones depende de la cantidad de fotones involucrados hablamos de fenómenos de \emph{Óptica Cuántica no lineal}.

\p El estudio de la interacción de medios atómicos no lineales con pocos fotones ha sido de gran interés en las últimas décadas, puesto que los estados no clásico de la luz son ``ladrillos'' para la creación de nuevas herramientas en la Óptica Cuántica y tecnologías en información cuántica~\cite{chen,gorniaczyk,tiarks}, donde queda en evidencia que la luz es excelente para transmitir información mientras que los átomos son un buen medio para procesarla. No obstante, alcanzar no linealidades a nivel de unos cuantos fotones es un reto experimental, por un lado es necesario que dichos fotones interactúen con un mismo átomo determinísticamente y luego que esto modifique el comportamiento del medio, es decir, que algunos átomos afecten todo el conjunto.

\p La probabilidad de interacción puede ser aumentada de dos formas: incrementando el número de fotones o agrandando la sección eficaz de interacción entre fotones y átomos. Lo primero es inconveniente cuando se desean hacer experimentos con fotones individuales, e incompatible si la intensidad de la radiación se vuelve suficientemente alta para producir no linealidades que se explican con modelos clásicos. La segunda opción es desafiante de conseguir en medios usuales, una forma de hacer más grande la sección eficaz es con cavidades de alta fineza, donde los fotones pasan varias veces por los átomos antes de abandonar la cavidad o extinguirse en sus extremos, esto produce una interacción coherente con los átomos~\cite{haroche, birnbaum}. Una desventaja es el tiempo que tarda en salir la luz de la cavidad, mientras que un resonador de baja fineza proporciona monitoreo rápido de su dinámica interna. Usando cavidades de baja fineza cuasi-concéntricas también aumentamos la probabilidad de interacción al reducir el área transversal del láser.

\p En el laboratorio de OCR donde desarrollo mi proyecto de doctorado, además de implementar una cavidad para tener fuertes interacciones entre átomos y fotones, incrementaremos la densidad atómica del medio y aprovecharemos las propiedades de los átomos de Rydberg: átomos con un electrón altamente excitado. Gracias a sus propiedades tan exageradas con el número cuántico principal $n$, la existencia de este tipo de átomos en el medio no sólo amplifica la sección eficaz de cada átomo~\cite{changde}, sino que provee una no linealidad propia al medio, el \emph{bloqueo de Rydberg}: la fuerte interacción entre átomos en estado de Rydberg provoca que exista una vecindad alrededor de los átomos en la cual no es posible excitar otro átomo a estado de Rydberg. La región de bloqueo, cuyo tamaño está en las micras, afecta el comportamiento de los miles de átomos dentro de dicha región.

\p Otro fenómeno a considerar para nuestros experimentos es la transparencia hacia una frecuencia de luz que se induce en los átomos debido a la presencia de un segundo haz de luz que produce interferencia cuántica en las amplitudes de las transiciones ópticas, lo que se denomina como Transparencia Electromagnéticamente Inducida (EIT, por sus siglas en inglés). Una nube de átomos en este escenario es un medio no lineal muy dispersivo que puede generar estados no clásicos de luz~\cite{zhu}. La idea de crear luz no clásica con el bloqueo de Rydberg se planteo en 2005~\cite{friedler}, pero no fue hasta la introducción de EIT que se lograron hacer detecciones no destructivas para medir las interacciones coherentes átomo luz~\cite{mohapatra,dudin,peyronel,maxwell}.

\p Estudiar y cuantificar los cambios que experimentan pulsos de luz que atraviesan átomos fríos en condiciones de EIT es mi proyecto de doctorado, lo cual servirá para caracterizar medios no lineales con los cuales produciremos estados no clásicos de la luz.