\chapter{\label{cap:objetivo}Objetivo general}

En el laboratorio de Óptica Cuántica de Rydberg (OCR), donde estoy realizando mi proyecto de doctorado, queremos crear estados de luz \emph{no clásica} utilizando medios no lineales con ayuda de átomos de Rydberg, lo que denominamos \emph{Óptica Cuántica no lineal de Rydberg}. Este tipo de átomos tienen un electrón excitado en un nivel de energía elevado, lo cual hace que su respuesta a campos externos sea mucho más intensa y de lugar a medios no lineales. Las propiedades de un haz o pulso de luz se modifican por las no linealidades de este tipo de medios, incluyendo la generación de estados no clásicos del campo electromagnético. De la interacción luz-materia podemos investigar, entre otras cosas, dos aspectos relacionados entre sí:

\begin{itemize}
\item[(a)] Las características de un medio a través de la luz.
\item[(b)] Cómo influye un medio conocido en la luz que lo atraviesa.
\end{itemize}

En el caso de (a) para poder ver o, en general, detectar un objeto, éste tiene que sobresalir de sus alrededores. Cuando un objeto destaca por su color o es detectado directamente por las frecuencias de la luz que refleja decimos que es un \emph{objeto de amplitud}, en caso contrario el objeto es transparente pues no refleja ni absorbe la radiación que incide sobre el mismo y lo nombramos \emph{objeto fase}. Lo anterior no excluye el hecho de que estos objetos fase sí modifican la luz, dicha interacción modifica la fase de la radiación que atraviesa el medio. Entonces, podemos obtener información de las propiedades y características de un medio fase a través del cambio de fase que induce el medio en la luz.

\p Mi proyecto de doctorado está centrado en caracterizar los cambios de fase que experimentan pulsos de luz láser debido a un medio fase altamente dispersivo, este medio será una nube de átomos fríos que es transparente a dichos pulsos gracias al fenómeno de \emph{Transparencia Electromagnéticamente Inducida}, que altera sus propiedades ópticas incluyendo el índice de refracción $n_\sm{r}$ y que nos permite tener control en el grado de dispersión $dn_\sm{r}/d\omega$. También medir y cuantificar el cambio en el ancho de los pulsos láser (no ultracortos) y su retraso debido a la dispersión. Además, estos experimentos se realizarán con átomos de Rydberg, con la perspectiva de poder utilizarlos en la generación de estados no clásicos de luz.

\p Caracterizar la luz servirá para, a través de esta, estudiar objetos fase. Conocer un objeto o medio fase tiene su importancia en diversas aplicaciones, tanto en su detección como en mediciones no destructivas donde intencionalmente se utiliza luz que no se absorbe por el medio, esto con la finalidad de no modificar sus propiedades, su dinámica interna o de alterar su estructura como puede ser una muestra biológica. Con mi proyecto espero contribuir al conocimiento en el comportamiento de una nube con átomos de Rydberg en un estado de Transparencia Electromagnéticamente Inducida, dicho medio atómico ha cobrado mucha relevancia en investigaciones de Óptica Cuántica no lineal y de Información Cuántica.