\chapter{\label{cap:objetivosEspecificos}Objetivo específicos}

En el capítulo anterior traté un poco la teoría alrededor de la Transparencia Electromagnéticamente Inducida (EIT), sobre los átomos de Rydberg y como son utilizados para producir medios atómicos con una respuesta no lineal dependiente del número de fotones, medios capaces de transformar estados de luz coherente o clásica a luz no clásica. La meta de mi doctorado es realizar experimentos que consisten en mandar pulsos de haces láser a través de un ensamble de átomos fríos de Rb en condiciones de EIT, medir y caracterizar estos pulsos, como el cambio en su fase o en su ancho. A través del estudio de la luz podremos conocer las propiedades de medios no lineales, ahí la importancia en el análisis de estos pulsos láser. También, aprovechando el fenómeno de luz lenta dentro de estos medios no lineales pretendemos conocer su forma y densidad sin modificar la dinámica interna de los átomos.

%\section{\label{sec:objetivoExperimento}Montaje y realización experimental}
%
%El montaje de la optomecánica necesaria lo realizaremos para poder mandar dos pulsos láser a través del medio en estado de EIT, uno de los pulsos será resonante con la ventana de transparencia del medio mientras que el otro estará desintonizado en frecuencia lo suficiente para que no sea absorbido por el medio. Una vez ambos pulsos láser atraviesen el medio se cambiará la frecuencia del pulso desintonizado para que ambos pulsos tengan la misma frecuencia y poder realizar una detección homodina (ver~\ref{sub:deteccionHomodina}) con el fin de conocer el cambio de fase introducido por el medio.
%
%\p Durante esta etapa optimizaremos la construcción de este sistema de cambio de frecuencias y detección para obtener una señal clara de un cambio de fase en la luz de entrada. Ya hemos hecho avances en una técnica para cambiar la frecuencia de uno de los dos pulsos láser y conocemos la fase relativa que se acumula entre ambos pulsos (ver~\ref{sec:cambiadorFase}).

\section{\label{sec:caracterizacionFase}Caracterización de los pulsos láser}

Cuantificaremos las propiedades de los pulsos láser que salen del medio no lineal, lo haremos utilizando el control que tendremos en lo dispersivo del medio, y por tanto en la velocidad de grupo de propagación de pulsos láser, a través de la intensidad del haz de control y la densidad atómica. Debido a la dispersión esperamos conocer el cambio en la fase de los pulsos y en su ancho, esto lo lograremos con la técnica de detección homodina (ver~\ref{sub:deteccionHomodina}). Además, contrastaremos con simulaciones de la propagación de tales pulsos en medios atómicos de esas características.

\section{\label{sec:medicionLuzLenta}Medición de luz lenta}

Utilizaremos fotodiodos de alta resolución ($\SI{150}{\mega\hertz}$) para medir el retraso que sufre un pulso láser debido a la nube de átomos altamente dispersiva. A través de esta luz lenta pretendemos caracterizar la nube atómica, puesto que la velocidad de grupo depende de la densidad medio, ésta se mapea al retraso de la luz. Al igual que con la fase, variando la intensidad del haz de control cambiamos el valor del retraso, utilizando la ecuación~\ref{ec:velocidadGrupo} conoceremos la densidad del medio y con~\ref{ec:retraso} su espesor, esto sin destruir la nube de átomos debido a procesos de absorción y emisión de fotones.